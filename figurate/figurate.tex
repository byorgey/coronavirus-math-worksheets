% -*- compile-command: "stack exec --package diagrams-lib --package diagrams-pgf --package diagrams-contrib --package diagrams-builder --package palette -- pdflatex --enable-write18 figurate.tex" -*-

\documentclass{article}

\usepackage{hyperref}
\usepackage{url}
\usepackage{amsmath}

\usepackage[outputdir=diagrams, extension=pgf, backend=pgf, input]{diagrams-latex}
\usepackage{pgf}

\usepackage{graphicx}
\graphicspath{{images/}}

%%%%%%%%%%%%%%%%%%%%%%%%%%%%%%%%%%%%%%%%%%%%%%%%%%%%%%%%%%%%

\title{Figurate Numbers and Forward Differences}
\author{Brent Yorgey, \href{http://www.mathlesstraveled.com}{\texttt{mathlesstraveled.com}} \\ \raisebox{-0.4em}{\includegraphics[width=44px]{../CC-BY.png}} \href{http://creativecommons.org/licenses/by/4.0/}{\texttt{creativecommons.org/licenses/by/4.0/}}}

\begin{document}

\maketitle

\fontsize{16}{20}\selectfont

To find the \emph{forward difference} of a list of numbers, we
subtract each number from the next one in the list.  For example, if
we start with this list:
\[
\begin{array}{ccccccccccccc}
  1 &   & 5 &   & 7 &   & 10 &   & 11 &   & 15 &   & 20 \\
\end{array}
\]
we first subtract $1$ from $5$, which gives $4$.  We write the $4$
underneath the list, in between the $1$ and the $5$, like this:
\[
\begin{array}{ccccccccccccc}
  1 &   & 5 &   & 7 &   & 10 &   & 11 &   & 15 &   & 20 \\
    & 4 &   &   &   &   &    &   &    &   &    &   &
\end{array}
\]
Next we subtract the $5$ from $7$ which gives us $2$.  We can write
the $2$ in between the $5$ and the $7$, like this:
\[
\begin{array}{ccccccccccccc}
  1 &   & 5 &   & 7 &   & 10 &   & 11 &   & 15 &   & 20 \\
    & 4 &   & 2 &   &   &    &   &    &   &    &   &
\end{array}
\]
If we keep going and write the difference of each pair of numbers in
between them, we get this:
\[
\begin{array}{ccccccccccccc}
  1 &   & 5 &   & 7 &   & 10 &   & 11 &   & 15 &   & 20 \\
    & 4 &   & 2 &   & 3 &    & 1 &    & 4 &    & 5 &
\end{array}
\]

Find the forward difference of each list of numbers.  The first
difference has been filled in for you.

\[ \arraycolsep=10pt
\begin{array}{ccccccccccccc}
  1 &   & 4 &   & 6 &   & 9 &   & 13 &   & 14 &   & 18 \\
    & 3 &   &   &   &   &   &   &    &   &    &   &
\end{array}
\] \vspace{1in}
\[
  \arraycolsep=10pt
\begin{array}{ccccccccccccc}
  1 &   & 3 &   & 5 &   & 7 &   & 9  &   & 11 &   & 13 \\
    & 2 &   &   &   &   &   &   &    &   &    &   &
\end{array}
\] \vspace{1in}
\[
  \arraycolsep=10pt
\begin{array}{ccccccccccccc}
  1 &   & 2 &   & 4 &   & 8 &   & 16 &   & 32 &   & 64 \\
    & 1 &   &   &   &   &   &   &    &   &    &   &
\end{array}
\] \vspace{1in}
\[
  \arraycolsep=10pt
\begin{array}{ccccccccccccc}
  1 &   & 2 &   & 3 &   & 5 &   & 8 &   & 13  &   & 21 \\
    & 1 &   &   &   &   &   &   &   &   &     &   &
\end{array}
\]

\newpage
Write down the number of dots in each triangle. Then find the forward
difference of the list of numbers that you get.

\begin{center}
\begin{diagram}[width=400]
dia = hsep 2 . map t $ [1 .. 7]

t :: Int -> Diagram B
t n = atPoints (rotateBy (1/6) edge) rows
  where
    edge :: [P2 Double]
    edge = fromOffsets . replicate (n-1) $ unitX
    rows :: [Diagram B]
    rows = map (hcat' (with & sep .~ 1 & catMethod .~ Distrib))
         . map (flip replicate pip) $ [n,n-1..1]

    pip = circle 0.2 # lw 0 # fc black
\end{diagram}
\end{center}

\newpage
Write down the number of dots in each square.  Then find the forward difference.

\begin{center}
\begin{diagram}[width=400]
dia = hsep 2 . map s $ [1 .. 7]

s :: Int -> Diagram B
s n = vcat' (with & sep .~ 1 & catMethod .~ Distrib)
  (replicate n (hcat' (with & sep .~ 1 & catMethod .~ Distrib) (replicate n pip)))
  # alignB
  where
    pip = circle 0.2 # lw 0 # fc black
\end{diagram}
\end{center}

\newpage
Write down the number of dots in each rectangle.  Then find the forward difference.

\begin{center}
\begin{diagram}[width=400]
dia = hsep 2 . map r $ [1 .. 6]

r :: Int -> Diagram B
r n = vcat' (with & sep .~ 1 & catMethod .~ Distrib)
  (replicate n (hcat' (with & sep .~ 1 & catMethod .~ Distrib) (replicate (n+2) pip)))
  # alignB
  where
    pip = circle 0.2 # lw 0 # fc black
\end{diagram}
\end{center}

\newpage
Questions for further reflection:
\begin{enumerate}
\item Taking the forward difference of a list of numbers produces
  another list of numbers.  What happens if we take the forward
  difference again?  And again?  And so on?  Try it on some of the
  example lists above.
\item Can you use your observations to predict the next triangle
  number?  The next square number?  The next $n \times (n+2)$
  rectangle number?
\end{enumerate}

\end{document}
