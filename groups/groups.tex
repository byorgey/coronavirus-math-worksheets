% -*- compile-command: "stack exec --package diagrams-lib --package diagrams-pgf --package diagrams-contrib --package diagrams-builder --package palette -- pdflatex --enable-write18 groups.tex" -*-
\documentclass{article}

\usepackage{hyperref}
\usepackage{url}
\usepackage{amsmath}

\usepackage[outputdir=diagrams, extension=pgf, backend=pgf, input]{diagrams-latex}
\usepackage{pgf}

\usepackage{graphicx}
\graphicspath{{images/}}

%%%%%%%%%%%%%%%%%%%%%%%%%%%%%%%%%%%%%%%%%%%%%%%%%%%%%%%%%%%%

\title{Group theory}
\author{Brent Yorgey, \href{http://www.mathlesstraveled.com}{\texttt{mathlesstraveled.com}} \\ \raisebox{-0.4em}{\includegraphics[width=44px]{../CC-BY.png}} \href{http://creativecommons.org/licenses/by/4.0/}{\texttt{creativecommons.org/licenses/by/4.0/}}}

\begin{document}

\maketitle

\fontsize{16}{20}\selectfont

Let's start by filling out some \emph{operation tables}.  Each table
will have some items listed across the top and side.  For a given
operation, your job is to fill each square with the result of that
operation on the items in the corresponding row and column.  For
example, if we list $2$, $3$, and $5$ along the side and top, and the
operation is addition, then the operation table would look like this:

\begin{center}
\begin{tabular}{l|lll}
$+$ & 2 & 3 & 5  \\ \hline
2   & 4 & 5 & 7  \\
3   & 5 & 6 & 8  \\
5   & 7 & 8 & 10
\end{tabular}
\end{center}

\section*{XOR}

Fill in each spot in the table with the XOR of the two Boolean values.

\begin{center}
\begin{tabular}{l|ll}
$\oplus$ & F & T \\ \hline
F        &   &   \\
T        &   &
\end{tabular}
\end{center}

\section*{Addition}

Fill in the table using addition.

\begin{center}
\begin{tabular}{l|lll}
$+$ & 0 & 1 & 2 \\ \hline
0        &   &   &   \\
1        &   &   &   \\
2        &   &   &
\end{tabular}
\end{center}

\section*{Addition modulo 3}

Fill in the table using \emph{addition modulo $3$}.  That is, imagine
that the number line ``wraps around'' back to $0$ when it gets to
$3$.  If you get an answer bigger than $3$, subtract $3$ from it until
you are back down under $3$.  For example, $2 +_3 2 = 1$ since $2 + 2
= 4$ and $4 - 3 = 1$.

\begin{center}
\begin{tabular}{l|lll}
$+_3$ & 0 & 1 & 2 \\ \hline
0        &   &   &   \\
1        &   &   &   \\
2        &   &   &
\end{tabular}
\end{center}

\section*{Addition modulo 5}

\begin{center}
\begin{tabular}{l|lllll}
$+_5$ & 0 & 1 & 2 & 3 & 4 \\ \hline
0        &   &   &   &   &   \\
1        &   &   &   &   &   \\
2        &   &   &   &   &   \\
3        &   &   &   &   &   \\
4        &   &   &   &   &  
\end{tabular}
\end{center}

\section*{Multiplication modulo 5}

Multiplication modulo $5$ works just like addition modulo $5$, but
with multiplication: to compute $a \times_5 b$, first multiply $a$ and
$b$, then keep subtracting $5$ until you get something smaller than
$5$.  For example, $3 \times_5 4 = 2$ since $3 \times 4 = 12$ and $12
- 5 - 5 = 2$.

\begin{center}
\begin{tabular}{l|llll}
$\times_5$ & 1 & 2 & 3 & 4 \\ \hline
1          &   &   &   &   \\
2          &   &   &   &   \\
3          &   &   &   &   \\
4          &   &   &   &
\end{tabular}
\end{center}

\newpage

Cut out the square on this page.  Ideally, you will also be able to
see the letter F through the back of the paper.  If you can't see it
very well you are welcome to draw in a backwards F on the back that
matches up with the F on the front. \vspace{1in}

\begin{center}
\begin{diagram}[width=300]
dia = text "F" # fontSizeL 0.25 <> square 1
\end{diagram}
\end{center}

\newpage

Let's list all the different things we can do to the square so
that it ends up still being a square in the same orientation (that is,
we want to list all the \emph{symmetries} of the square).  This might
seem familiar from reading the book \emph{Beautiful Symmetry}.

\end{document}
