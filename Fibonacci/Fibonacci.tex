% -*- compile-command: "stack exec --package diagrams-lib --package diagrams-pgf --package diagrams-contrib --package diagrams-builder --package palette -- pdflatex --enable-write18 Fibonacci.tex" -*-
\documentclass{article}

\usepackage{hyperref}
\usepackage{url}
\usepackage{amsmath}

\usepackage[outputdir=diagrams, extension=pgf, backend=pgf, input]{diagrams-latex}
\usepackage{pgf}

\usepackage{graphicx}
\graphicspath{{images/}}

%%%%%%%%%%%%%%%%%%%%%%%%%%%%%%%%%%%%%%%%%%%%%%%%%%%%%%%%%%%%

\title{Fibonacci numbers}
\author{Brent Yorgey, \href{http://www.mathlesstraveled.com}{\texttt{mathlesstraveled.com}} \\ \raisebox{-0.4em}{\includegraphics[width=44px]{../CC-BY.png}} \href{http://creativecommons.org/licenses/by/4.0/}{\texttt{creativecommons.org/licenses/by/4.0/}}}

\begin{document}

\maketitle

\fontsize{16}{20}\selectfont

The \emph{Fibonacci numbers} are defined by
\begin{align*}
  F_0 &= 0 \\
  F_1 &= 1 \\
  F_n &= F_{n-1} + F_{n-2} & (n > 1)
\end{align*}

That is, the $0$th Fibonacci number is $0$; the $1$st Fibonacci number
is $1$; and every number after that is formed by adding the two
previous numbers.  The first 17 Fibonacci numbers are therefore
\[ 0,1,1,2,3,5,8,13,21,34,55,89,144,233,377,610,987. \]

\begin{enumerate}
\item What is the next Fibonacci number?
  \newpage

  \item Fill in the blanks below.
    \begin{align*}
      F_0 + F_1 &= \underline{\phantom{XXX}} \\[0.25in]
      F_0 + F_1 + F_2 &= \underline{\phantom{XXX}} \\[0.25in]
      F_0 + F_1 + F_2 + F_3 &= \underline{\phantom{XXX}} \\[0.25in]
      F_0 + F_1 + F_2 + F_3 + F_4 &= \underline{\phantom{XXX}} \\[0.25in]
    \end{align*}
  \item Keep going for three more steps. \vspace{2in}
  \item Do you notice a pattern?
\newpage
  \item If $F_{28} = 317811$ and $F_{29} = 514229$, what is $F_{27}$? \vspace{1in}
  \item In general, if you know $F_n$ and $F_{n-1}$, how can you find
    $F_{n-2}$? \vspace{1in}
  \item Using your answer to the previous questions, what should
    $F_{-1}$ be (that is, the Fibonacci number before $F_0$)?
    \vspace{1in}
  \item What about $F_{-2}$? \newpage
  \item Write down the first ten Fibonacci numbers that should come
    before $0$, that is, $F_{-1}$ through $F_{-10}$. \vspace{3in}
  \item What pattern do you notice?
    \newpage
    \newcommand{\m}{\mathrm{m}} There is a path that is $1$ meter wide
    and $n$ meters long, and you want to pave it with stones.  You
    have two kinds of paving stones: square dark grey stones that are
    $1\m \times 1\m$ and rectangular light grey stones that are
    $1\m \times 2\m$.

    \begin{center}
    \begin{diagram}[width=150]
import Fibonacci

dia = hsep 2 [s1, s2]
    \end{diagram}
    \end{center}

    This means you might have some different
    choices in terms of how you pave your path.  For example, if the
    path is $2$ meters long, you have two choices: you could pave it
    with a single $1 \times 2$ stone, or you could pave it with two
    $1\times 1$ stones.

    \begin{center}
    \begin{diagram}[width=200]
import Fibonacci

dia = (vsep 2 . map centerX $
  [ path 2 # centerXY # named "path"
  , hsep 2 [paving [2] # centerX # named "p2", paving [1,1] # centerX # named "p11"]
  ]
  )
  # applyAll
    [ connectOutside' (with & gap .~ local 0.5) "path" "p2"
    , connectOutside' (with & gap .~ local 0.5) "path" "p11"
    ]
    \end{diagram}
    \end{center}
\newpage
\item If the path is $3$ meters long, how many different ways do you
  have to pave it?  Draw them here. \vspace{2in}

\item What if the path is $4$ meters long? \newpage

\item What about $5$ meters long? \vspace{3in}

\item Do you notice a pattern?  Use the pattern to predict how many
  ways there are to pave a path that is $15$ meters long.

  \newpage

Let's say a sequence of bits is \emph{invalid} if it has two $0$ bits
right next to each other, and \emph{valid} otherwise.  For example,
$1110100101$ is invalid, since it has $00$ in the middle.  On the
other hand, $1110110101$ is valid since none of the $0$s are right
next to each other.

\item Is $110111010$ valid or invalid? \vspace{1in}
\item What about $11111$? \vspace{1in}
\item Write down all the valid sequences of two bits.  How many are there? \vspace{2.5in}
\item Write down all the valid sequences of three bits.  How many are
  there? \vspace{2in}
\item Write down all the valid sequences of four bits.  How many are
  there? \vspace{2in}
\item Do you notice a pattern?  How many valid sequences of $10$ bits
  do you think there will be?

  \newpage

\end{enumerate}

\end{document}
